\documentclass[12pt,a4paper]{article}
\usepackage[utf8]{inputenc}
\usepackage[brazilian]{babel}
\usepackage[T1]{fontenc}
\usepackage{amsmath}
\usepackage{amsfonts}
\usepackage{amssymb}
\usepackage{url}
\usepackage{listings}

\title{\Large UFMG/ICEx/DCC\\
\Large Compiladores I\\
\large Professora: Mariza Andrade da Silva Bigonha
\normalsize 2º Semestre de 2014\\
}
\author{Pablo Marcondes Fonseca\\
\texttt{pablom@dcc.ufmg.br}}

\begin{document}

%\maketitle
\centerline{\Large UFMG/ICEx/DCC}
\medskip
\centerline{\Large Compiladores I}
\medskip
\centerline{\large Professora: Mariza Andrade da Silva Bigonha}
\medskip
\centerline{\normalsize 2º Semestre de 2014}
\medskip
\centerline{\large Aluno: Pablo Marcondes Fonseca \texttt{pablom@dcc.ufmg.br}}
\bigskip
\centerline{\Large \textbf{Analisador Sintático}}

\section{Descrição}

O analisador sintático foi implementado na linguagem C++, utilizando a ferramenta bison. O bison foi utilizado para gerar o parser a partir de uma notação própria, onde inserimos a gramática da linguagem e o mesmo gera um parser utilizando tabelas LALR.

A maior dificuldade encontrada foi com relação ao uso da ferramenta, principalmente na escrita da gramática em outra notação. A partir da leitura de trechos do manual oficial, alguns códigos de exemplo, e vários testes, foi possível entender o funcionamento do sistema de geração de parser e criar o arquivo adequado para concluir o trabalho.


\section{Implementação e Resultados}

Foi criado um programa principal para executar continuamente o analisador sintático sobre o arquivo de entrada, esse programa tem seu código na listagem \ref{lst:main}

\lstinputlisting[basicstyle=\footnotesize,language=C++,caption={Programa principal},label={lst:main}]{test.cpp}

\medskip
O analisador sintático utilizando a ferramenta Bison para gerar código, é criado a partir da especificação em uma linguagem própria do Bison (sucessor do Yacc), é uma notação muito próxima de C. No arquivo parser.yy, que pode ser visto na listagem \ref{lst:parser}, definiu-se um conjunto de \textit{tokens} que são retornados pelo analisador léxico ao encontrar-se cada uma das partes definidas na gramática utilizada. O parser verifica se as construções estão de acordo com a gramática, e retorna erro caso encontra alguma expressão que não possa ser derivada dela a partir do símbolo inicial.

\lstinputlisting[basicstyle=\footnotesize,language=C++,caption={Programa Yacc},label={lst:parser}]{parser.yy}

\medskip
O método utilizado para se verificar a correção do analisador sintático, assim como no caso do analisador léxico, foi a execução do programa sobre cada um dos testes disponibilizados na especificação do trabalho prático. Os programas de teste utilizados, bem como os resultados de execução seguem listados. Nos resultados, há a impressão de todos os tokens identificados até que aconteça algum erro, facilitando a identificação do ponto com problema:

\lstinputlisting[basicstyle=\footnotesize,language=pascal,caption={Programa teste1.lsm},label={lst:in1}]{teste1.lsm}

\lstinputlisting[basicstyle=\scriptsize,language=pascal,caption={Saída para teste1.lsm},label={lst:out1}]{teste1.lsm.par}


\lstinputlisting[basicstyle=\footnotesize,language=pascal,caption={Programa teste2.lsm},label={lst:in2}]{teste2.lsm}

\lstinputlisting[basicstyle=\scriptsize,language=pascal,caption={Saída para teste2.lsm},label={lst:out2}]{teste2.lsm.par}


\lstinputlisting[basicstyle=\footnotesize,language=pascal,caption={Programa teste3.lsm},label={lst:in3}]{teste3.lsm}

\lstinputlisting[basicstyle=\scriptsize,language=pascal,caption={Saída para teste3.lsm},label={lst:out3}]{teste3.lsm.par}

\pagebreak
\lstinputlisting[basicstyle=\footnotesize,language=pascal,caption={Programa teste4.lsm},label={lst:in4}]{teste4.lsm}

\lstinputlisting[basicstyle=\scriptsize,language=pascal,caption={Saída para teste4.lsm},label={lst:out4}]{teste4.lsm.par}


\lstinputlisting[basicstyle=\footnotesize,language=pascal,caption={Programa teste5.lsm},label={lst:in5}]{teste5.lsm}

\lstinputlisting[basicstyle=\scriptsize,language=pascal,caption={Saída para teste5.lsm},label={lst:out5}]{teste5.lsm.par}


\lstinputlisting[basicstyle=\footnotesize,language=pascal,caption={Programa teste6.lsm},label={lst:in6}]{teste6.lsm}

\lstinputlisting[basicstyle=\scriptsize,language=pascal,caption={Saída para teste6.lsm},label={lst:out6}]{teste6.lsm.par}


\lstinputlisting[basicstyle=\footnotesize,language=pascal,caption={Programa teste7.lsm},label={lst:in7}]{teste7.lsm}

\lstinputlisting[basicstyle=\scriptsize,language=pascal,caption={Saída para teste7.lsm},label={lst:out7}]{teste7.lsm.par}

\end{document}
