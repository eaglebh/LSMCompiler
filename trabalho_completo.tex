\documentclass[12pt,a4paper]{article}
\usepackage[utf8]{inputenc}
\usepackage[brazilian]{babel}
\usepackage[T1]{fontenc}
\usepackage{amsmath}
\usepackage{amsfonts}
\usepackage{amssymb}
\usepackage{url}
\usepackage{listings}

\lstset{inputencoding=latin1} 

\title{\Large UFMG/ICEx/DCC\\
\Large Compiladores I\\
\large Professora: Mariza Andrade da Silva Bigonha
\normalsize 2º Semestre de 2014\\
}
\author{Pablo Marcondes Fonseca\\
\texttt{pablom@dcc.ufmg.br}}

\begin{document}

%\maketitle
\centerline{\Large UFMG/ICEx/DCC}
\medskip
\centerline{\Large Compiladores I}
\medskip
\centerline{\large Professora: Mariza Andrade da Silva Bigonha}
\medskip
\centerline{\normalsize 2º Semestre de 2014}
\medskip
\centerline{\large Aluno: Pablo Marcondes Fonseca \texttt{pablom@dcc.ufmg.br}}
\bigskip
\centerline{\Large \textbf{Trabalho Prático Completo}}


\section{Introdução}

O trabalho foi realizado utilizando-se a linguagem C++ em boa parte do código, as ferramentas flex e bison foram essenciais para a criação do compilador. A linguagem intermediária escolhida foi a MEPA, por sua simplicidade e suporte ao que era necessário para implementação de um compilador para a linguagem LSM.

Foram entregues relatórios parciais para a Tabela de Símbolos, Analisador Léxico e Analisador Sintático. Houve alterações no código desses elementos do compilador, neste relatório são explicitadas as diferenças principais dessas três partes, e o código completo é listado ao final. Não há neste relatório a repetição dos resultados do que já foi entregue, pois a execução final já compreende todas essas etapas. 

Há detalhes maiores para as etapas seguintes, como a execução final já compreende todas as partes do compilador, toda a listagem é feita no final do relatório.

\section{Tabela de Símbolos}

A tabela de simbolos implementada anteriormente utilizava basicamente código em C, na implementação melhorada utilizou-se a linguagem C++ para facilitar alguns aspectos da implementação, como manipulação da pilha e de lista auxiliar. Tanto a pilha quanto a lista utilizam apontadores para reduzir o uso de memória RAM, compartilhando entre si os símbolos apontados.

A implementação atual chama-se SymbolStack, e possui todos os métodos básicos que permitem Instalar, informar Entrada de Bloco, Saída de Bloco, Recuperar símbolo, entre outras operações. Utilizou-se as estruturas de dados stack e list da STL do C++ para implementação da tabela de símbolos. Parte do código pode ser visualizado nas listagens \ref{lst:sth} e \ref{lst:stcpp}

\lstinputlisting[basicstyle=\footnotesize,language=C++,caption={Tabela de símbolos - cabeçalho},label={lst:sth}]{SymbolStack.h}

\lstinputlisting[basicstyle=\footnotesize,language=C++,caption={Tabela de símbolos - corpo},label={lst:stcpp}]{SymbolStack.cpp}


\section{Analisador Léxico}

O analisador léxico sofreu poucas modificações, apenas refatoração de algumas partes e desabilitou-se o método que imprimia a saída a cada leitura realizada, então o resultado final é praticamente o mesmo do inicial.

O código final pode ser visto na listagem \ref{lst:lex}, com poucas diferenças da primeira versão já entregue.

\lstinputlisting[basicstyle=\footnotesize,language=C++,caption={Programa Lex},label={lst:lex}]{scanner.ll}


\section{Analisador Sintático}

Quanto à análise sintática não houve muitas modificações, apenas o necessário para resolver alguns conflitos criados com a geração de código. Para encontrar os problemas habilitou-se a saída de debug do bison, dessa forma foi possível identificar os estados com conflitos de shift-reduce e reduce-reduce criados, o que permitiu a modificação do compilador a fim de resolver esses problemas.

Como o código final contém a parte de análise sintática, verificação de tipos e também geração de código, há uma listagem apenas compreendendo todas as partes implementadas no arquivo de entrada da ferramenta bison, o parser.yy que pode ser visualizado na seção \ref{listagem}.

\section{Análise Semântica}

A análise semântica compreendendo tradução e verificação de tipos foi realizada sobre o arquivo de entrada para o bison, o parser.yy. Criou-se um conjunto de métodos auxiliares para geração e verificação de tipos, além de algumas variáveis de controle.

\section{Código Intermediário}

\section{Execução}


\section{Listagem Final de Códigos} \label{listagem}

\lstinputlisting[basicstyle=\footnotesize,language=C++,caption={Programa Yacc},label={lst:parser}]{parser.yy}

\medskip
O método utilizado para se verificar a correção do analisador sintático, assim como no caso do analisador léxico, foi a execução do programa sobre cada um dos testes disponibilizados na especificação do trabalho prático. Os programas de teste utilizados, bem como os resultados de execução seguem listados. Nos resultados, há a impressão de todos os tokens identificados até que aconteça algum erro, facilitando a identificação do ponto com problema:

\lstinputlisting[basicstyle=\footnotesize,language=pascal,caption={Programa teste1.lsm},label={lst:in1}]{t/teste1.lsm}

\lstinputlisting[basicstyle=\scriptsize,language=pascal,caption={Saída para teste1.lsm},label={lst:out1}]{t/teste1.lsm.out}


\lstinputlisting[basicstyle=\footnotesize,language=pascal,caption={Programa teste2.lsm},label={lst:in2}]{t/teste2.lsm}

\lstinputlisting[basicstyle=\scriptsize,language=pascal,caption={Saída para teste2.lsm},label={lst:out2}]{t/teste2.lsm.out}


\lstinputlisting[basicstyle=\footnotesize,language=pascal,caption={Programa teste3.lsm},label={lst:in3}]{t/teste3.lsm}

\lstinputlisting[basicstyle=\scriptsize,language=pascal,caption={Saída para teste3.lsm},label={lst:out3}]{t/teste3.lsm.out}

\pagebreak
\lstinputlisting[basicstyle=\footnotesize,language=pascal,caption={Programa teste4.lsm},label={lst:in4}]{t/teste4.lsm}

\lstinputlisting[basicstyle=\scriptsize,language=pascal,caption={Saída para teste4.lsm},label={lst:out4}]{t/teste4.lsm.out}


\lstinputlisting[basicstyle=\footnotesize,language=pascal,caption={Programa teste5.lsm},label={lst:in5}]{t/teste5.lsm}

\lstinputlisting[basicstyle=\scriptsize,language=pascal,caption={Saída para teste5.lsm},label={lst:out5}]{t/teste5.lsm.out}


\lstinputlisting[basicstyle=\footnotesize,language=pascal,caption={Programa teste6.lsm},label={lst:in6}]{t/teste6.lsm}

\lstinputlisting[basicstyle=\scriptsize,language=pascal,caption={Saída para teste6.lsm},label={lst:out6}]{t/teste6.lsm.out}


\lstinputlisting[basicstyle=\footnotesize,language=pascal,caption={Programa teste7.lsm},label={lst:in7}]{t/teste7.lsm}

\lstinputlisting[basicstyle=\scriptsize,language=pascal,caption={Saída para teste7.lsm},label={lst:out7}]{t/teste7.lsm.out}

\end{document}
